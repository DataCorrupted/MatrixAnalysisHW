 
\documentclass{article}

\title{Matrix Analysis Homework 7}
\author{Rong Yuyang \\ Student ID: 69850764 \\ rongyy@shanghaitech.edu.cn}

\usepackage[utf8]{inputenc}
\usepackage{graphicx}
\usepackage[colorlinks,linkcolor=red]{hyperref}
\usepackage{amsmath, amsthm, amssymb}
\usepackage{subfloat}
\newtheorem{prop}{Proposition}
\usepackage{ulem}
\usepackage{indentfirst}
\begin{document}
\maketitle

\begin{description}
	\item[Problem 1] (Problem Description)
	.\\
	\textbf{i) Dimension}
		$$dim(\mathcal{W}) = s(m+n-s)$$
		\begin{proof}
			let's take $\mathcal{W}$ apart:
			$$\mathcal{W} = \mathcal{W}_U + \mathcal{W}_V$$
			where we have:
			$$ \mathcal{W}_U = \{u_ix^T | u_i \in U, x \in \mathbb{R}^n\}$$
			$$ \mathcal{W}_V = \{yv_j^T | v_j \in V, y \in \mathbb{R}^m\}$$
			we can extend basis of $U$, $V$ so that they span the whole $\mathbb{R}^m$, $\mathbb{R}^n$, respectively:
			$$U' = U + \{u'_i\}; V' = V + \{v'_i\}$$
			then we can write the basis of $ \mathcal{W}_U$, $\mathcal{W}_V$ as:
			$$B_{\mathcal{W}_U} = \{u_iv_j^T\} + \{u_iv'_j^T\}$$
			$$B_{\mathcal{W}_V} = \{u_iv_j^T\} + \{u'_iv_j^T\}$$
			finally:
			\begin{equation}\begin{aligned}
				\dim(\mathcal{W}) 
				& = \dim(\mathcal{W}_U + \mathcal{W}_V) \\
				& = |B_{\mathcal{W}_U} + B_{\mathcal{W}_V}| \\
				& = |\{u_iv_j^T\} + \{u_iv'_j^T\} + \{u'_iv_j^T\}| \\
				& = s^2 + s(n-s) + s(m-s) \\
				& = s(m+n-s)
			\end{aligned}\end{equation} 
		\end{proof}
	\textbf{ii) Projection}
		$$\forall W \in \mathcal{W}$$
		$$W_P = P_UWP_V$$
	\textbf{iii) Oth Proj}
		$$W_O = W - W_P$$

	\item[Problem 2] (Problem Description)
	.\\
	\textbf{i) U}
	$$\forall \lambda \in \sigma(U), |\lambda| = 1$$
	\begin{proof}
		Suppose we have eigenpair $(\lambda, v)$:
		$$ Uv = \lambda v$$
		$$ v^*U^* = v^*\lambda^*$$
		$$ v^*U^* Uv = v^*v = v^*\lambda^*\lambda v$$
		$$ v^*v = \lambda^*\lambda v^*v$$
		$$ \lambda^*\lambda = |\lambda| = 1$$
	\end{proof}
	\textbf{ii) s}
	\begin{proof}
		Take eigenpair: $\forall (\lambda, v)$
		$$P^2v = P\lambda v = \lambda^2 v$$
		$$ \lambda = 0, 1 \Rightarrow \sigma(\lambda) = {0, 1}$$
		For eigenvalue 1, the corresponding eigenvectors form the basis of subspace $S$.\\
		Then we have:
		$$ \mu_a = k; \mu_g = n - k$$
	\end{proof}

\end{description}

\end{document}
