 
\documentclass{article}

\title{Matrix Analysis Homework 9}
\author{Rong Yuyang \\ Student ID: 69850764 \\ rongyy@shanghaitech.edu.cn}

\usepackage[utf8]{inputenc}
\usepackage{graphicx}
\usepackage[colorlinks,linkcolor=red]{hyperref}
\usepackage{amsmath, amsthm, amssymb}
\usepackage{subfloat}
\newtheorem{prop}{Proposition}
\usepackage{ulem}
\usepackage{indentfirst}
\begin{document}
\maketitle

\begin{description}
	\item[Problem 1] (Problem Description)
	\begin{proof}
		Let's keep the definition of $A_{1:j}$ the same as we talked in the class:
		$$ A_{1:j} = \sum_{s = 1}^{j}\lambda_s(A)\mu_s\mu_s^T$$
		Now we have:
		\begin{equation}\begin{aligned}
			\lambda_{j+k-n}(A+B)
			& \geq \lambda_{n}(A-A_{j+1:n} + B-B_{k+1:n}) \\
			& \geq \lambda_{n}(A-A_{j+1:n}) + \lambda_{n}(B-B_{k+1:n}) \\
			&   =  \lambda_{j}(A) + \lambda_{k}(B)
		\end{aligned}\end{equation}
	\end{proof}

	\item[Problem 2] (Problem Description)
	\begin{proof}
		\begin{equation}\begin{aligned}
			\lambda_k(A)
			& = \min_{\dim(V) = n-k+1} \max_{||x||_2 =1} x^TAX  \\
			& \geq \min_{\dim(V) = n-k+1} \max_{||x||_2 = 1, x \perp \{e_{1}, \cdots e_{i}\}} x^TAx \\
			& = \min_{\dim(V\cap\{e_{1}, \cdots e_{i}\}) = n-k+1} \max_{||x||_2 = 1} x^TB'x \\
			& = \lambda_k(B)


		\end{aligned}\end{equation}
	\end{proof}

	\item[Problem 3] (Problem Description)
	\begin{proof}

		Here goes your anwser.

	\end{proof}
\end{description}

\end{document}
