 
\documentclass{article}

\title{Matrix Analysis Homework 9}
\author{Rong Yuyang \\ Student ID: 69850764 \\ rongyy@shanghaitech.edu.cn}

\usepackage[utf8]{inputenc}
\usepackage{graphicx}
\usepackage[colorlinks,linkcolor=red]{hyperref}
\usepackage{amsmath, amsthm, amssymb}
\usepackage{subfloat}
\newtheorem{prop}{Proposition}
\usepackage{ulem}
\usepackage{indentfirst}
\usepackage[a4paper,includeheadfoot,margin=2.54cm]{geometry}

\begin{document}
\maketitle

\begin{description}
	\item[Problem 1] (Problem Description)
	\begin{proof}
		Let's keep the definition of $A_{1:j}$ the same as we talked in the class:
		$$ A_{1:j} = \sum_{s = 1}^{j}\lambda_s(A)\mu_s\mu_s^T$$
		Now we have:
		\begin{equation}\begin{aligned}
			\lambda_{j+k-n}(A+B)
			& \geq \lambda_{n}(A-A_{j+1:n} + B-B_{k+1:n}) \\
			& \geq \lambda_{n}(A-A_{j+1:n}) + \lambda_{n}(B-B_{k+1:n}) \\
			&   =  \lambda_{j}(A) + \lambda_{k}(B)
		\end{aligned}\end{equation}
	\end{proof}

	\item[Problem 2] (Problem Description)
	\begin{proof}
		\begin{equation}\begin{aligned}
			\lambda_k(A)
			& = \min_{\dim(V) = n-k+1} \max_{||x||_2 =1} x^TAX  \\
			& \geq \min_{\dim(V) = n-k+1} \max_{||x||_2 = 1, x \perp \{e_{1}, \cdots e_{i}\}} x^TAx \\
			& = \min_{\dim(V\cap\{e_{1}, \cdots e_{i}\}) = n-k+1} \max_{||x||_2 = 1} x^TB'x \\
			& (wlog) = \min_{V'} \max_{x = \begin{bmatrix}0 \\ z\end{bmatrix}} \begin{bmatrix}0 \\ z\end{bmatrix}\begin{bmatrix} &  \\  & B\end{bmatrix}\begin{bmatrix}0 & z\end{bmatrix}
			= \min \max_{||z||_2 = 1} z^TBz \\
			& = \lambda_k(B)
		\end{aligned}\end{equation}
	\end{proof}

	\item[Problem 3] (Problem Description)
	\begin{proof}
		Given $U$, extend it to $V = \begin{bmatrix}U & Q\end{bmatrix} \Rightarrow V^TV = I_n$. \\
		Thus $V^TAV$ is similar to $A \Rightarrow \lambda_k(A) = \lambda_k(V^TAV)$
		\begin{equation}
			V^TAV = \begin{bmatrix}Q & U_{k\times k}\end{bmatrix}^TA\begin{bmatrix}Q & U\end{bmatrix} = \begin{bmatrix} & \\ & U^TAU\end{bmatrix}
		\end{equation}
		By interlacing II we have:
		\begin{equation}\begin{aligned}
			& \lambda_i (U^TAU) \geq \lambda_n+i-k(V^TAV) = \lambda_{n-k+i}(A) \\
			& \Leftrightarrow tr(U^TAU) = \sum_{i = 1}^k \lambda_i(U^AU) \leq \sum_{i}\sigma_{i}(A)
		\end{aligned}\end{equation}
		Last, the equal condition: 
		\par With $(u_i, \lambda_i(A))$ being eigenpairs where $\lambda_1 > \lambda_2 > \cdots$.
		When $U = \begin{bmatrix}u_1 & u_2 & \cdots & u_k\end{bmatrix}$

	\end{proof}
	\item[Problem 4] (Problem Description)
	\begin{proof}
		(Just a example of Interlacing II)
		$$\forall i, B_{1 \times 1} = \begin{bmatrix}A_{ii}\end{bmatrix} \Rightarrow \lambda_1(B) = A_{ii}$$
		Now consider interlancing II:
		\begin{equation}\begin{aligned}
			& \lambda_{n+1-1}(A) \leq \lambda_1(B) \leq \lambda_{1}(A) \\
			& \Leftrightarrow \lambda_n(A) \leq A_{ii} \leq \lambda_{1}(A)
		\end{aligned}\end{equation}
		\end{proof}
	\item[Problem 5] (Problem Description)
	\begin{proof}
		\begin{equation}\begin{aligned}
			    \sum_{i = 1}^k\lambda_i(A+B) 
			& =	\max_{\forall U_{k \times k}} tr(U^T(A+B)U) = \max_{\forall U_{k \times k}} tr(U^TAU+U^TBU) \\ 
			& = \max_{\forall U_{k \times k}} tr(U^TAU)+tr(U^TBU) \\
			& \leq \max_{\forall U_{k \times k}} tr(U^TAU)+\max_{\forall U_{k \times k}} tr(U^TBU) \\
			& = \sum_{i = 1}^k\lambda_i(A) + \sum_{i = 1}^k\lambda_i(B) 
		\end{aligned}\end{equation}
	\end{proof}

\end{description}

\end{document}
